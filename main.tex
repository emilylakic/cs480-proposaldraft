\documentclass[
	12pt, % Default font size, values between 10pt-12pt are allowed
	%letterpaper, % Uncomment for US letter paper size
	%spanish, % Uncomment for Spanish
]{fphw}

% Template-specific packages
\usepackage[utf8]{inputenc} % Required for inputting international characters
\usepackage[T1]{fontenc} % Output font encoding for international characters
\usepackage{mathpazo} % Use the Palatino font

\usepackage{graphicx} % Required for including images

\usepackage{booktabs} % Required for better horizontal rules in tables

\usepackage{listings} % Required for insertion of code

\usepackage{enumerate} % To modify the enumerate environment

\title{CS480 Project Proposal Draft} % Assignment title

\author{Emily Lakic and Umut Kayaalti} % Students name

\date{September 26th, 2019} % Due date

\institute{Binghamton University \\ Thomas J. Watson School of Engineering} % Institute or school name

\class{Intelligent Mobile Robotics (CS480R/CS580R)} % Course or class name

\professor{Dr. Shiqi Zhang} % Professor or teacher in charge of the assignment


\begin{document}

\maketitle % Output the assignment title, created automatically using the information in the custom commands above


\section*{Proposal Draft Topic}

\begin{center}
	\includegraphics[width=0.5\columnwidth]{robotsfollowhuman-450x218.jpg} % Example image
\end{center}


\subsection*{Answer}

Robotic Vision.

\section*{Our Idea}

\subsection*{Answer}

Our robot will actively follow a person while keeping a certain distance. If that person moves away, the robot will move until the person is as close as the specified distance again, or will rotate to keep him/her in its lenses.

\section*{Purpose}

\subsection*{Answer} 

A human-following robot has plenty of applications in daily life and manufacturing. Autonomous robots can learn to 'follow the leader' in an effort to provide as mission partners for humans, whether to assist in day-by-day tasks or for more extensive use as industrial robots operating in complex environments.

\section*{Past Research}

\subsection*{Answer}

 Human-following robots have been researched and developed actively in the past few decades. Previous research has explored several techniques such as human’s target detection, robot control algorithm, and obstacles avoidance. Various approaches have been followed, including the use of ultrasonic, voice recognition, and laser range sensors, as well as a charge-coupled device (CCD) camera, and so on. We will explore these possibilities in our research and development of the final project.

\end{document}
